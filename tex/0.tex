\documentclass{article}
\usepackage{amsmath}

\begin{document}

\section{4. CONSERVATION LAWS}

\subsection{4.2. Density in a Horizontal Flow}

The density in a horizontal flow $\mathbf{u} = U(y,z)\mathbf{e}_x$ is given by $\rho(\mathbf{x}, t) = f(x - Ut, y, z)$, where $f\left(x, y, z\right)$ is the density distribution at $t=0$. Is this flow incompressible?

Letting $\xi = x - Ut$, we have:

$$
\frac{D\rho}{Dt} = \frac{\partial\rho}{\partial t} + \mathbf{u} \cdot\nabla\rho = \frac{\partial\rho}{\partial\xi}\frac{\partial\xi}{\partial t} + U\frac{\partial\rho}{\partial\xi}\frac{\partial\xi}{\partial x} = \frac{\partial\rho}{\partial\xi}(-U) + U\frac{\partial\rho}{\partial\xi}(1) = 0.
$$

Additionally, we have:

$$
\nabla\cdot\mathbf{u} = \frac{\partial U(y,z)}{\partial x} + 0 + 0 = 0.
$$

Therefore, this is an incompressible flow, but the density may vary when $f$ is not constant.

\subsection{4.3 STREAM FUNCTIONS}

Consider the steady form of the continuity equation (4.7):

The divergence of the curl of any vector field is identically zero (see Exercise 2.21), so $\rho\mathbf{u}$ will satisfy (4.11) when written as the curl of a vector potential $\Psi$:

$$
\rho\mathbf{u} = \nabla\times\Psi,
$$

where $\Psi = \chi\nabla\psi$. Furthermore, $\rho\mathbf{u} = \nabla\chi\times\nabla\psi$, because the curl of any gradient is identically zero. Also, $\nabla\chi$ is perpendicular to surfaces of constant $\chi$, and $\nabla\psi$ is perpendicular to surfaces of constant $\psi$. Therefore, the mass flux $\rho\mathbf{u} = \nabla\chi\times\nabla\psi$ is also perpendicular to these surfaces.

Letting $\chi=a$, $\chi=b$, $\psi=c$, and $\psi=d$, we have:

The mass flux $\dot{m}$ through the surface $A$ bounded by the four stream surfaces (shown in gray in Figure 4.1) is calculated with area element $dA$, normal n (as shown), and Stokes' theorem.

Defining the mass flux $\dot{m}$ through $A$, and using Stokes' theorem produces:

$$
\begin{aligned}
\dot{m} &= \int\rho\mathbf{u}\cdot\mathbf{n}\ dA = \int_A (\nabla\times\Psi)\cdot\mathbf{n}\ dA \\
&= \int_C \Psi\cdot d\mathbf{s} = \int_C \chi\nabla\psi\cdot d\mathbf{s} \\
&= \int_C \chi d\psi \\
&= b(d-c) + a(c-d) \\
&= (b-a)(d-c).
\end{aligned}
$$

\end{document}