\documentclass{article}
\usepackage{amsmath}

\begin{document}

\section{4.2 Conservation of Mass}

\subsection{The Continuity Equation}

The final equality can only be possible if the integrand vanishes at every point in space. If the integrand did not vanish at every point in space, then integrating (\ref{eq:4.6}) in a small volume around a point where the integrand is nonzero would produce a nonzero integral. Thus, (\ref{eq:4.6}) requires:

\begin{equation}
\frac{\partial\rho(\mathbf{x}, t)}{\partial t} + \nabla\cdot(\rho\mathbf{u}(\mathbf{x}, t)) = 0 \quad\text{or, in index notation: } \frac{\partial\rho}{\partial t} + \frac{\partial}{\partial x_i}(\rho u_i) = 0.\tag{4.7}
\end{equation}

This relationship is called the continuity equation. It expresses the principle of conservation of mass in differential form, but is insufficient for fully determining flow fields because it is a single equation that involves two field quantities, $\rho$ and $\mathbf{u}$, and $\mathbf{u}$ is a vector with three components.

\subsection{Divergence of Mass-Density Flux}

The second term in (\ref{eq:4.7}) is the divergence of the mass-density flux $\rho\mathbf{u}$. Such flux divergence terms frequently arise in conservation statements and can be interpreted as the net loss at a point due to divergence of a flux. For example, the local $\rho$ will decrease with time if $\nabla\cdot(\rho\mathbf{u})$ is positive. Flux divergence terms are also called transport terms because they transfer quantities from one region to another without making a net contribution over the entire field. When integrated over the entire domain of interest, their contribution vanishes if there are no sources at the boundaries.

\subsection{Alternative Form of the Continuity Equation}

The continuity equation may alternatively be written using the definition of $D/Dt$ (\ref{eq:3.5}) and $\partial(\rho u_i)/\partial x_i = u_i \partial\rho/\partial x_i + \rho \partial u_i / \partial x_i$ [see (\ref{B3.6})]:

\begin{equation}
\frac{1}{\rho(\mathbf{x}, t)}\frac{D}{Dt}\rho(\mathbf{x}, t) + \nabla\cdot\mathbf{u}(\mathbf{x}, t) = 0.\tag{4.8}
\end{equation}

The derivative $D\rho/Dt$ is the time rate of change of fluid density following a fluid particle. It will be zero for constant density flow where $\rho=$ constant throughout the flow field, and for incompressible flow where the density of fluid particles does not change but different fluid particles may have different density:

\begin{equation}
\frac{D\rho}{Dt} \equiv \frac{\partial\rho}{\partial t} + \mathbf{u}\cdot\nabla\rho = 0.\tag{4.9}
\end{equation}

Taken together (\ref{eq:4.8}) and (\ref{eq:4.9}) imply:

\begin{equation}
\nabla\cdot\mathbf{u} = 0.\tag{4.10}
\end{equation}

for incompressible flows. Constant density flows are a subset of incompressible flows; $\rho=$ constant is a solution of (\ref{eq:4.9}) but it is not a general solution. A fluid is usually called incompressible if its density does not change with pressure. Liquids are almost incompressible. Gases are compressible, but for flow speeds less than $-100\text{ m/s}$ ($\mathrm{i.s.},$ for Mach numbers <0.3) the fractional change of absolute pressure in a room temperature airflow is small. In this and several other situations, density changes in the flow are also small and (\ref{eq:4.9}) and (\ref{eq:4.10}) are valid.

\subsection{General Form of the Continuity Equation}

The general form of the continuity equation (\ref{eq:4.7}) is typically required when the derivative $D\rho/Dt$ is nonzero because of changes in the pressure, temperature, or molecular composition of fluid particles.

\end{document}