\documentclass{article}
\usepackage{amsmath}

\begin{document}

\section{Finite-state machines 3}

\subsection{The Tap-Light Finite State Machine}

\begin{center}
\includegraphics[width=0.5\textwidth]{figure1}
\caption{The tap-on/tap-off table light}
\label{fig:tap-light}
\end{center}

\section{Introduction}
A finite state machine (FSM) is a mathematical model that can be in one of a finite number of states. The tap-light FSM is an example of such a model.

\subsection{The Diagram}
In the diagram, the two circles, named \emph{o f f} and on, represent the state of the light. The state of the light can be changed by the occurrence of an event, named tap.

\begin{center}
  \includegraphics[width=0.5\textwidth]{figure2}
  \caption{The tap-light in the $O F F$ state}
  \label{fig:tap-light-fff}
\end{center}

\subsection{Transitions}
We show a transition from one state to another by a directed arc (A line with an arrow on one end) from one state to another, indicating the starting state of the diagram. The transitions are as follows:

$\begin{array}{rll} & \text{o f f}\xrightarrow{\text{tap}}&\text{on}\\ &\text{on}\xrightarrow{\text{tap}}&\text{o f f}\end{array}$

\subsection{Behavior}
The FSM can be shown by putting a dot inside the current state. For example, immediately after starting this FSM, the diagram will appear as shown in fig. 3.

$\begin{center} \includegraphics[width=0.5\textwidth]{figure3} \caption{The tap-light in the on state} \label{fig:tap-light-on} \end{center}$

All this behavior can be understood just by putting your finger on the current state (the one with the dot) and, upon receipt of an event, following the appropriate transition to the next state.

\subsection{Current State}
Consider our previous diagram, with the tap-light in the $O F F$ state. If we receive a tap event, the diagram changes to the on state, as shown in fig. 4.

$\begin{center} \includegraphics[width=0.5\textwidth]{figure4} \caption{The tap-light in the on state} \label{fig:tap-light-on-2} \end{center}$

\end{document}