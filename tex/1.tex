\documentclass{article}
\usepackage{amsmath}

\begin{document}

\section{CONSERVATION OF MASS}

\subsection{The Continuity Equation}

The final equality can only be possible if the integrand vanishes at every point in space. If the integrand did not vanish at every point in space, then integrating (4.6) in a small volume around a point where the integrand is nonzero would produce a nonzero integral. Thus, (4.6) requires:

\[
\frac{\partial\rho({\bf x}, t)}{\partial t} + \nabla\cdot(\rho({\bf x}, t){\bf u}({\bf x}, t)) = 0 \quad\mathrm{or}, \quad\mathrm{i n ~ i n d e x ~ n o t a t i o n}: \quad\frac{\partial\rho}{\partial t} + \frac{\partial}{\partial {\bf x}_i}(\rho u_i) = 0. \tag{4.7}
\]

This relationship is called the continuity equation. It expresses the principle of conservation of mass in differential form, but is insufficient for fully determining flow fields because it is a single equation that involves two field quantities, $\rho$ and ${\bf u}$, and ${\bf u}$ is a vector with three components.

The second term in (4.7) is the divergence of the mass-density flux $\rho{\bf u}$. Such flux divergence terms frequently arise in conservation statements and can be interpreted as the net loss at a point due to divergence of a flux. For example, the local $\rho$ will decrease with time if $\nabla\cdot(\rho{\bf u})$ is positive. Flux divergence terms are also called transport terms because they transfer quantities from one region to another without making a net contribution over the entire field. When integrating (4.7) over the entire domain of interest, their contribution vanishes if there are no sources at the boundaries.

The continuity equation may alternatively be written using the definition of $D/Dt$ and $\nabla\cdot({\bf u}{\bf j})/\partial {\bf x}$:

\[
\frac{1}{D}\frac{\partial\rho({\bf x}, t)}{\partial t} = \frac{\partial\rho({\bf x}, t)}{\partial {\bf x}_i}\frac{{\bf u}_i + V - \rho\frac{\partial {\bf u}}{\partial {\bf x}_i}}{1}. \tag{4.8}.
\]

The derivative $D\rho/Dt$ is the time rate of change of fluid density following a fluid particle. It will be zero for constant density flow where $\rho$ = constant throughout the flow field, and it will be non-zero otherwise.

Taken together, (4.8) and the definition of $D/Dt$ imply:

\[
\frac{\partial\rho}{\partial t} = u_i \frac{\partial\rho}{\partial x_i} + \rho \frac{\partial {\bf u}_i}{\partial x_i}. \tag{4.9}
\]

or $\rho$ = constant is a solution of (4.9) but it is not a general solution. A fluid is usually called incompressible if its density does not change with pressure. Liquids are almost incompressible. Gases are compressible, but for flow speeds less than $\sim\! 100 \, \mathrm{m/s}$ (that is, for Mach numbers <0.3) the fractional change of absolute pressure in a room temperature airflow will be small.

The derivative $D\rho/Dt$ is also equal to:

\[
\frac{\partial\rho}{\partial t} = \rho \frac{\partial {\bf u}}{\partial t}. \tag{4.10}
\]

This equation shows that the time rate of change of fluid density is directly proportional to the mass flux, which is a measure of the amount of fluid flowing through a given area per unit time.

\end{document}