\documentclass{article}
\usepackage{amsmath}

\begin{document}

\section*{Finite-state machines 5}

\subsection*{Alphabets}

An alphabet is a finite, non-empty set of symbols. It is conventional to use the Greek letter $\sigma$ (Z) to represent an alphabet. Some examples of common alphabets are:

1. $\Sigma = \{0, 1 \}$ , the set of binary digits.
2. $\Sigma = \{A, B, \cdots, Z \}$ , the set of Roman letters.
3. $\Sigma = \{N, E, S, W \}$ , the set of compass-points.

\subsection*{Strings}

A string is a finite sequence of symbols drawn from an alphabet. A string is also sometimes called a word. Some examples of strings are:

1. $100101$ is a string from the binary alphabet $\Sigma = \{0, 1 \}$.
2. $THEORY$ is a string from the Roman alphabet $\Sigma = \{A, B, \cdots, Z \}$.

\subsection*{Empty String}

The empty string is a string with no symbols in it, usually denoted by the Greek letter $\epsilon$. Clearly, the empty string is a string that can be chosen from any alphabet.

\subsection*{Length of a String}

It is handy to classify strings by their length, the number of symbols in the string. The string $THEORY$ , for example, has a length of 6 . The usual notation for the length of a string $\mathcal{S}$ is $| s |$ . Thus $| T H E O R Y |=6$ , $1001 =4$, and $|\epsilon|=0$.

\subsection*{Powers of an Alphabet}

We are often interested in the set of all strings of a certain length, say $k,$ drawn from an alphabet $\Sigma$. This can be constructed by taking the Cartesian product, of $\Sigma$ with itself $k$ times: $\Sigma \times \Sigma \times \cdots \Sigma.$

For the alphabet $\Sigma = \{N, E, S, W\}$ , we find:

$$
\Sigma^1 = \{\mathrm{N}, \mathrm{E}, \mathrm{S}, \mathrm{W} \}
$$

$$
\Sigma^2 = \{\mathrm{N N}, \mathrm{N E}, \mathrm{N S}, \mathrm{N W}, \mathrm{E N}, \mathrm{E E}, \mathrm{E S}, \mathrm{E W}, \mathrm{S N}, \mathrm{S E}, \mathrm{S S}, \mathrm{S W}, \mathrm{W N}, \mathrm{W E}, \mathrm{W S}\}
$$

$\Sigma^3$ has 64 members, since it contains $4 \times4 \times4$ members.

The set of all strings that can be drawn from an alphabet is conventionally denoted, using the so-called Kleene star, by $\Sigma^{*}$ , and $\mathrm{of course}$ has an infinite number of members. For the alphabet $\Sigma = \{0, 1\}
$$
\Sigma^{*}=\{\epsilon, 0, 1, 00, 01, 10, 11\}
$$

Clearly, $\Sigma^{*}=\Sigma^{0} \cup\Sigma^{1} \cup\Sigma^{2} \cup\cdots$

\end{document}