\documentclass{article}
\usepackage{amsmath}

\begin{document}

\section{Conservation Laws}

\subsection{Mass Conservation}

Based on (\ref{eq:4.1}), the principle of mass conservation clearly constrains the fluid density. The implications of (\ref{eq:4.1}) for the fluid velocity field may be better displayed by using Reynolds transport theorem (\ref{eq:3.35}) with $F=\rho$ and ${\bf b}={\bf u}$ to expand the time derivative in (\ref{eq:4.1}):
$$
\int_{V ( t )} \frac{\partial\rho( \mathbf{x}, t )} {\partial t} \, d V+\int_{A ( t )} \rho( \mathbf{x}, t ) \mathbf{u} ( \mathbf{x}, t ) \cdot\mathbf{n} \, d A \,=\, 0. \tag{42}
$$
This is a mass-balance statement between integrated density changes within $V ( t )$ and integrated motion of its surface $A ( t )$. Although general and correct, (\ref{eq:4.2}) may be hard to utilize in practice because the motion and evolution of $V ( t )$ and $A ( t )$ are determined by the flow, which may be unknown.

To develop the integral equation that represents mass conservation for an arbitrarily moving control volume $V^{*} ( t )$ with surface $A^{*} ( t )$, (\ref{eq:4.2}) must be modified to involve integrations over $V^{*} ( t )$ and $A^{*} ( t )$. This modification is motivated by the frequent need to conserve mass within a volume s not a material volume, for example a stationary control volume.

The first step in this modification is to set $F=\rho$ in (\ref{eq:3.35}) to obtain:
$$
\frac{d} {d t} \, \int_{V^{*} ( t )} \! \rho( {\bf x}, t ) d V-\! \int_{V^{*} ( t )} \, \frac{\partial\rho( {\bf x}, t )} {\partial t} d V-\! \int_{A^{*} ( t )} \! \rho( {\bf x}, t ) {\bf b} \cdot{\bf n} \, d A \,=\, 0.
$$

The second step is to choose the arbitrary control volume $V^{*} ( t )$ to be instantaneously coincident with material volume $V ( t )$ so that at the moment of interest $V ( t )=V^{*} ( t )$ and $A ( t )=A^{*} ( t )$. At this coincidence moment, the $(d/dt)$ pdV-terms in (\ref{eq:4.1}) and (\ref{eq:4b}) are different.

\begin{align*}
\int_{V ( t )} \frac{\partial\rho( \mathbf{x}, t )} {\partial t} \, d V+\int_{A ( t )} \rho( \mathbf{x}, t ) \mathbf{u} ( \mathbf{x}, t ) \cdot\mathbf{n} \, d A \,=\, 0.
\end{align*}

and

\begin{align*}
\int_{V ( t )} \frac{\partial\rho( \mathbf{x}, t )} {\partial t} \, d V+\int_{A ( t )} \rho( \mathbf{x}, t ) \mathbf{u} ( \mathbf{x}, t ) \cdot\mathbf{n} \, d A \,=\, \int_{V ( t )} \! \left\{\frac{\partial\rho( \mathbf{x}, t )} {\partial t}+\nabla\! \cdot\! \left( \rho( \mathbf{x}, t ) \mathbf{u} ( \mathbf{x}, t ) \right) \right\} d V \,=\, 0.
\end{align*}

In particular, when ${\bf b}={\bf u}$,

The two equations for the same system are different in their surface integrals.

\end{document}