\documentclass{article}
\usepackage{amsmath}

\begin{document}

\section{CONSERVATION LAWS}
\subsection{EXAMPLE 4.2}

The density in a horizontal flow $\mathbf{u}=U(y,z)\mathbf{e}_x$ is given by $\rho(\mathbf{x}, t)=f(x-Ut,y,z)$, where $f(x,y,z)$ is the density distribution at $t=0$. Is this flow incompressible?

\begin{proof}
There are two ways to answer this question. First, consider (4.9) and evaluate $\frac{D\rho}{Dt}$, letting $\xi=x-Ut$:
$$
\frac{D\rho}{Dt}=\frac{\partial\rho}{\partial t}+\mathbf{u}\cdot\nabla\rho=\frac{\partial\rho}{\partial t}+U\frac{\partial\rho}{\partial x}=\frac{\partial\rho}{\partial\xi}\frac{\partial\xi}{\partial t}+U\frac{\partial\rho}{\partial\xi}\frac{\partial\xi}{\partial x}=\frac{\partial\rho}{\partial\xi}(-U)+U\frac{\partial\rho}{\partial\xi}(1)=0.
$$
Second, consider (4.10) and evaluate $\nabla\cdot\mathbf{u}$:
$$
\nabla\cdot\mathbf{u}=\frac{\partial U(y,z)}{\partial x}+0+0=0.
$$
This is an incompressible flow, but the density may vary when $f$ is not constant.

\end{proof}

\section{STREAM FUNCTIONS}

Consider the steady form of the continuity equation (4.7):
$$
\nabla\cdot(\rho\mathbf{u}) ~=~ 0. \tag{4.11}
$$

The divergence of the curl $\mathrm{o f}$ any vector field is identically zero (see Exercise 2.21), so $\rho\mathbf{u}$ will satisfy (4.11) when written as the curl of a vector potential $\Psi$ :
$$
\rho\mathbf{u} ~=~ \nabla\times\mathbf{\Psi}, \tag{4.12}
$$
for $\Psi$ which can be specified in terms of two scalar functions: $\Psi=\chi\nabla\psi.$  Putting this specification into (4.12) produces $\rho\mathbf{u}=\nabla\chi\times\nabla\psi,$ because the curl of any gradient is identically zero (see Exercise 2.22). Furthermore, $\nabla\chi$ is perpendicular to surfaces of constant $\rho\mathbf{u}=\nabla\chi\, \times\, \nabla\psi$, and $\nabla\psi$ is perpendicular to surfaces of constant $\psi$. Therefore, three-dimensional streamlines are the intersections of the two stream surfaces, or stream functions in a three-dimensional flow.

The situation is illustrated in Figure 4.1. Consider two members of each of the families of the two stream functions $\chi=a, \, \chi=b, \psi=c, \psi=d$ . The intersections shown as darkened lines in Figure 4.1 are the streamlines. The mass flux $\dot{m}$ through the surface $A$ bounded by the four stream surfaces (shown in gray in Figure 4.1) is calculated with area element $d A$ ,normal $\mathbf{n}$ (as shown), and Stokes' theorem.

Defining the mass flux $\dot{m}$ through $A,$ and using Stokes' theorem produces:
$$
\begin{aligned} {\dot{m}} & {{} \,=\, \int\rho\mathbf{u}\cdot\mathbf{n} \, d A \,=\, \int_{A} \, ( \nabla\times\mathbf{\Psi} ) \cdot\mathbf{n} \, d A \,=\, \int_{C} \mathbf{\Psi}\cdot d \mathbf{s} \,=\, \int_{C} \chi\nabla\psi\cdot d \mathbf{s} \,=\, \int_{C} \chi d \psi} \\ {} & {{} \,=\, b(d-c)+a(c-d) \,=\, (b-a)(d-c). \end{aligned}
$$}

\end{document}