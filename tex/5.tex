\documentclass{article}
\usepackage{amsmath}

\begin{document}

\section*{Introduction}
Sometimes we do not want to include the empty string in the set. The set of non-empty strings is denoted by $\Sigma^{+}$. This is often referred to as the Kleene plus, by analogy with the Kleene star.

Clearly, $\Sigma^{+}=\Sigma^{1} \cup\Sigma^{2} \cup\Sigma^{3} \cup\cdots$ 
And $\Sigma^{*}=\{\epsilon\} \cup\Sigma^{+}$ 

Concatenating strings
Let $\mathcal{S}$ be the string composed of the $m$ symbols $s_{1} s_{1} s_{2} \cdots s_{m},$ and $t$ be the string composed of the  $n$ symbols $t_{1} t_{1} t_{2} \cdots t_{n}$ . The concatenation of the strings $\mathcal{S}$ and $t$, denoted by $s t,$ is the string of length $m+n,$ composed of the symbols $s_{1} s_{1} s_{2} \cdots s_{m} t_{1} t_{1} t_{2} \cdots t_{n}.$ 

It is clear that the string $\epsilon$ can be concatenated with any other string $\mathcal{S}$ and that: $\epsilon\mathcal{S}=\mathcal{S}\epsilon=\mathcal{S}$. $\epsilon$ thus behaves as the identity value, for concatenation.

\section*{Languages}

A set of strings, all of which have been chosen from $\Sigma^{*}$ of an alphabet $\Sigma,$ is called a language. If $L \subseteq\Sigma^{*}$ , then $L$ is said to be a language over $\sum$ .

A language over $\sum$ does not need to include strings with  symbols of . The implication of this is that when we know that $L$ is a language over $\Sigma$ then $L$ is also a language over any alphabet that is a superset of $\sum$ .

The use of the word language" here is entirely consistent with everyday usage. For example the language "English" can be considered to be a set of strings drawn from the alphabet of Roman letters.

\section*{Examples}

1. The language consisting of valid binary byte-values (a string of 8 0's or 1's). This is just $\Sigma^{8}$ .

2. The set of even-parity binary numbers (having an even number of 1's), whose first digit is a 
$\begin{array}{l}
11, 101, 110,1001, 1010, 1100.1111.
\end{array}$ 

3. The set of valid compass directions: 
$\begin{align*}
N.S, E, W, NE, NW SE, SW, NNE, ENE,
\end{align*}$
4. $\Sigma^{*}$ is a language over an alphabet $\sum$ .

5. $\{\epsilon\}$ , the language consisting only of the empty string, is a language over any alphabet. This
language has just one string: $\epsilon$ 

6. $\O,$ the language with nostrings, is a language over any alphabet. Note that $\emptyset\neq\{\epsilon\},$ because 
$\begin{array}{l}
\{\epsilon\} \mbox{ contains one string}.
\end{array}$

Notice also that an alphabet $\sum$ is always of a finite size, but a language over that alphabet can either be of finite or of infinite size.

\end{document}