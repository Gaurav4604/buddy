\documentclass{article}
\usepackage[utf8]{inputenc}
\usepackage[T1]{fontenc}
\usepackage{amsmath}

\title{Mathematical Definitions in Formal Language Theory}
\author{}
\date{}

\begin{document}

\section*{Introduction}
Sometimes we do not want to include the empty string in the set. The set of non-empty strings is denoted by $\Sigma^{+}$. This is often referred to as the Kleene plus, by analogy with the Kleene star.

Clearly $\Sigma^{+}=\Sigma^{1} \cup\Sigma^{2} \cup\Sigma^{3} \cup\cdots$ 
And $\Sigma^{*}=\{\epsilon\} \cup\Sigma^{+}$

Concatenating strings
Let $s$ be the string composed of the $m$ symbols $s_{1} s_{1} s_{2} \cdots s_{m},$ and $t$ be the string composed of the $n$ symbols $t_{1} t_{1} t_{2} \cdots t_{n}$ . The concatenation of the strings $s$ and $t,$ denoted by $st$, is the string of length $m+n,$ composed of the symbols $s_{1} s_{1} s_{2} \cdots s_{m} t_{1} t_{1} t_{2} \cdots t_{n}$ .

It is clear that the string $\epsilon$ can be concatenated with any other string $s$ and that: $es = se = s$.  Thus, it behaves as the identity value for concatenation.

\section{Languages}

A set of strings, all of which have been chosen from $\Sigma^{*}$ of an alphabet $\Sigma,$ is called a language. If $L \subseteq\Sigma^{*}$ , then $L$ is said to be a language over $\Sigma$ .

A language over ${\Sigma}$ does not need to include strings with  all the symbols of $\Sigma$. The implication of this is that when we know that $L$ is a language over $\Sigma,$ then $L$ is also a language over any alphabet that is a superset of $\Sigma$.

The use of the word “language” here is entirely consistent with everyday usage. For example, the language "English" can be considered to be a set of strings drawn from the alphabet of Roman letters.

The programming language Java, or indeed any other programming language, is another example. The set of syntactically-correct programs is the set of strings that can be formed from the alphabet of the language (the ASCII characters).

Using the alphabets we defined earlier, we can specify some languages that might be of interest to us:

1. The language consisting of valid binary byte-values (a string of 8 or 13):
$00000001 \in \Sigma^{8}$
2. The set of even-parity binary numbers (having an even number of 1's), whose first digit is a $1$: 
$11.1.101001.10.1011.+3$
3. The set of valid compass directions: $(N, S, E, W, NE, NW, SE, SW)$.
4. $\Sigma^{*}$ is a language over an alphabet $\Sigma$.

5. $\{\epsilon\}$ , the language consisting only of the empty string, is a language over any alphabet. This language has just one string: $\epsilon$.

Note also that an alphabet $\Sigma$ is always of a finite size, but a language over that alphabet can either be of finite or infinite size.

\end{document}