\documentclass{article}
\usepackage{amsmath}

\begin{document}

\section{Conservation of Mass}

\subsection*{Example 4.1}

In isothermal liquid flows, the fluid density is typically a known constant. What are the dependent field variables in this case? How many equations are needed for a successful mathematical description of such flows? What physical principles supply these equations?

\subsubsection*{Solution}

When the fluid’s temperature is constant and its density is a known constant, the thermal energy of fluid elements cannot be changed by heat transfer or work because \(dT = dv = 0\), so the thermodynamic characterization of the flow is complete from knowledge of the density. Thus, the dependent field variables are \( \mathbf{u} \), the fluid’s velocity (momentum per unit mass), and the pressure, \( p \). Here, \( p \) is not a thermodynamic variable; instead, it is a normal force (per unit area) developed between neighboring fluid particles that either causes or results from fluid-particle acceleration, or arises from body forces. 

Thus, four equations are needed: one for each component of \( \mathbf{u} \), and one for \( p \). These equations are supplied by the principle of mass conservation and three components of Newton’s second law for fluid motion (conservation of momentum).

\section{Conservation of Mass}

Setting aside nuclear reactions and relativistic effects, mass is neither created nor destroyed. Thus, individual mass elements — molecules, grains, fluid particles, etc. — may be tracked within a flow field because they will not disappear and new elements will not spontaneously appear. The equations representing conservation of mass in a flowing fluid are based on the principle that the mass of a specific collection of neighboring fluid particles is constant.

The volume occupied by a specific collection of fluid particles is called a material volume \( V(t) \). Such a volume moves and deforms within a fluid flow so that it always contains the same mass elements; none enter the volume and none leave it. This implies that a material volume’s surface \( A(t) \), a material surface, must move at the local fluid velocity \( \mathbf{u} \) so that fluid particles inside \( V(t) \) remain inside and fluid particles outside \( V(t) \) remain outside. Thus, a statement of conservation of mass for a material volume in a flowing fluid is:

\begin{equation}
\frac{d}{dt} \int_{V(t)} \rho(\mathbf{x}, t) \, dV = 0,
\end{equation}

where \( \rho \) is the fluid density. Figure 3.20 depicts a material volume when the control surface velocity \( \mathbf{b} \) is equal to \( \mathbf{u} \). The primary concept here is equivalent to an infinitely flexible, perfectly sealed thin-walled balloon containing fluid. The balloon’s contents play the role of the material volume \( V(t) \) with the balloon itself defining the material surface \( A(t) \). And, because the balloon is sealed, the total mass of fluid inside the balloon remains constant as the balloon moves, expands, contracts, or deforms.

\end{document}
